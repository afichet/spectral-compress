\documentclass[tikz]{standalone}
\usepackage{tikz}
\usepackage{pgfplots}
\usetikzlibrary{spy}
\usetikzlibrary{calc}
\usetikzlibrary{datavisualization}
\usetikzlibrary{shapes.misc}
\usetikzlibrary{math}

\begin{document}
\tikzset{cross/.style={cross out, draw=black, minimum size=2*(#1-\pgflinewidth), inner sep=0pt, outer sep=0pt},
%default radius will be 1pt.
cross/.default={1pt}}

\begin{tikzpicture}[spy using outlines={magnification=5, size=1.5cm}]
    \def \leftScale{0}
    \def \rightScale{0.5}
    \def \topScale{5}
    \def \bottomScale{0}
    \tikzmath{
        \scaleWidth  = \rightScale - \leftScale;
        \scaleHeight = \topScale - \bottomScale;
    }

    \def \scaleStart{-5}
    \def \scaleEnd{0}

    \foreach \y in {-4,...,-1} {
        \tikzmath{
            \uY = (\y - \scaleStart) / (\scaleEnd - \scaleStart);
            \posY = \uY * (\topScale - \bottomScale) + \bottomScale;
        }
        \draw (-.1, \posY) -- (0, \posY);
        \node[left, inner sep=0pt] at (-.15, \posY) {$10^{\y}$};
    }

    % Shift a bit top & bottom
    \draw (-.1, \bottomScale) -- (0, \bottomScale);
    \node[above left, inner sep=0pt] at (-.15, \bottomScale) {$10^{-5}$};

    \draw (-.1, \topScale) -- (0, \topScale);
    \node[below left, inner sep=0pt] at (-.15, \topScale) {$10^{0}$};

    \draw (\leftScale, \topScale) -- (\rightScale, \topScale) -- (\rightScale, \bottomScale) -- (\leftScale, \bottomScale) -- cycle;

    \node[anchor=north west, inner sep=0] at (\leftScale, \topScale)
        {\includegraphics[width=\scaleWidth cm, height=\scaleHeight cm]{scale.png}};

\end{tikzpicture}

\end{document}
