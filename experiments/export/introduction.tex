\section{Introduction}

In this supplemental, we benchmark our compression technique using the two spectral image databases described in the paper.
We convert from spectral OpenEXR to our spectral JPEG XL using a variety of parameters.
We compare to a simple compression scheme, which saves each spectral band independently using JPEG-XL compression.


\subsection{Spectral OpenEXR Conversion}

Our compression pipeline uses spectral OpenEXR files holding 32-bit floats as input and outputs a collection of JPEG-XL files.
For the CAVE database we get a set of grayscale PNGs with 16-bit unsigned integers for each spectral band.
Using spectral OpenEXR with 32-bit floats, we avoid loss of precision in the conversion.
The conversion is implemented in the supplemental program \verb?cave-exr? and orchestrated by the script \verb?experiment/cave_to_exr.py?.

The Bonn database is originally provided in the \verb?.npz? Numpy format.
We export the diffuse and specular maps as spectral OpenEXR using 32-bit floats per band.
We use the script \verb?experiment/bonn_to_exr.py? to execute this conversion.

All spectral OpenEXR files use ZIP compression.
Our reported compression ratios are relative to the file sizes of these spectral OpenEXR files.
For the sake of a fair comparison, the spectral OpenEXR files do not contain RGB layers.


\subsection{Compression Parameters}

We vary the following parameters to explore their impact on quality, compression ratios and timings:

\begin{description}
    \item[Distance level] This parameter controls the image error metric used by JPEG XL and thus the compression ratio. When the value is 0, we use lossless compression. The admissible range is 0 to 15. Higher values decrease image quality and file size. For our spectral JPEG XL, the distance level for the DC component and the first AC component are controlled by the user. We use two parameter choices here:
    \begin{itemize}
        \item A distance level of 0 for the DC component (lossless) and a distance level of 1 for the first AC component (lossy),
        \item A distance level of 0.5 for the DC component (lossy) and a distance level of 2 for the first AC component (lossy).
    \end{itemize}
    \item[Compression curve] As described in the paper, we have three ways of defining a compression curve (i.e. the distance levels for higher-frequency AC components):
    \begin{itemize}
        \item Flat curve: All AC components use the same distance level,
        \item Deterministic curve: The distance level grows from the prescribed value to 15 according to a hard-coded sigmoid curve,
        \item Dynamic curve: The distance levels are determined for each frequency in an automated fashion such that they all introduce similar error.
    \end{itemize}
    \item[Chroma subsampling] We also support subsampling of AC components while keeping the DC component at full resolution. Since most of the chroma information is held by the normalized AC components, we name this ``chroma subsampling''. We use two settings:
        \begin{itemize}
            \item No chroma subsampling (1:1), i.e. all frequencies are kept at full resolution,
            \item Chroma subsampling (1:2), i.e. the DC component is kept at full resolution while the resolution of all AC components is halved (both horizontally and vertically).
        \end{itemize}
\end{description}

For each run, the JPEG-XL effort parameter is set to 7, which is the default value in libjxl.


\subsection{Metrics}

For each test of a compression method, we report the following metrics.

\paragraph{Error}

For RGB images, perceptual metrics are preferable but for spectral images there are no established standards for such metrics.
Therefore, we fall back to reporting simple root-mean-square errors (RMSEs) per pixel and per image.
The RMSE for a single pixel $x,y$ is defined as
\begin{equation}
    \mathrm{RMSE}(g,g^{\prime},x,y):=\sqrt{\frac{\sum_{l=0}^{n-1}(g_{x,y,l}^{\prime}-g_{x,y,l})^{2}}{n}} \text{,}
\end{equation}
where $g_{x,y,l}$ is the ground truth image, $g^\prime_{x,y,l}$ is the decompressed image and the images have width $w$, height $h$ and $n$ spectral bands.
We use a color map to display these RMSEs for each pixel.
The RMSE across the whole images is
\begin{equation}
    \mathrm{RMSE}(g,g^{\prime}):=\sqrt{\frac{\sum_{x=0}^{w-1}\sum_{y=0}^{h-1}\sum_{l=0}^{n-1}(g_{x,y,l}^{\prime}-g_{x,y,l})^{2}}{whn}} \text{.}
\end{equation}


\paragraph{Compression Ratio}

A compression ratio is the file size of the spectral OpenEXR file (with ZIP compression and with RGB layers removed) divided by the size of the file collection for the spectral JPEG XL (or the collection of JPEG XL files for individual spectral bands in case of the simple method).


\paragraph{Timings}

All computations were done on an Intel Core i9-10980XE CPU with 128~GB of RAM (Dell Precision 5820 Tower X-Series).
On each image, we provide the timings for the execution of the whole compression pipeline.
For the summary section, we instead provide the average timing per pixel since the image resolution varies.


\paragraph{Previews}

We provide an sRGB version of the spectral OpenEXR images as preview.
The exposure is set to 0 for the Bonn database and to -6.5 for the CAVE database.
We also show an inset of the RGB versions and the diff images of the $50\times50$~pixels in the middle of the image to illustrate compression artifacts of different lossy approaches.

\clearpage
